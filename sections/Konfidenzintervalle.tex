\section{Konfidenzintervalle}
\begin{mainbox}{Definition Konfidenzintervall}
	Sei \(\alpha \in [0,1]\). Ein Konfidenzintervall für \(\theta\) mit Niveau \(1 - \alpha\) ist ein Zufallsintervall \(I(\omega)=[A(\omega),B(\omega)]\), sodass gilt
	\[\forall \theta \in \Theta \quad \P_\theta[A\le \theta \le B] \ge 1- \alpha\]

	wobei \(A\) und \(B\) Zufallsvariablen der Form \(A = a(X_1, \ldots, X_n), B = b(X_1, \ldots, X_n)\) mit \(a,b: \R^n \to \R\) sind.
\end{mainbox}

Wenn wir einen Schätzer \(T = T_{ML} \sim \mathcal{N}\left(\theta, \text{Var}(X)\right)\) haben, suchen wir ein Konfidenzintervall der Form
\[I = \left[T-c \sqrt{\Var(X)}, T+c\sqrt{\Var(X)}\right]\]
Hierbei gilt:
\begin{align*}
	\P_\theta&\left(T-c\sqrt{\Var(X)} \le \theta \le T+c\sqrt{\Var(X)}\right) \\
	&= \P_\theta\left(-c\le Z \le c\right)\\
	&= \P_{\theta}\left(Z \leq c\right) - \P_{\theta}\left(Z < -c\right)\\
	&= \P_{\theta}\left(Z \leq c\right) - \left(1-\P_{\theta}\left(Z \leq c\right)\right)\\
	&= 2\Phi(c)-1
\end{align*}
wobei \(Z = \frac{T-\theta}{\sqrt{\Var(X)}} \sim \mathcal{N}(0,1)\) ist.


\subsection{Approximatives Konfidenzintervall}
Wir können den zentralen Grenzwertsatz benutzen, um eine standardnormalverteilte ZV zu erhalten, und damit die Konfidenzintervalle zu bestimmen.

